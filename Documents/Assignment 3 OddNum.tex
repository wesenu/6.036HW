\documentclass[12pt]{amsart}
\usepackage{amssymb}

\usepackage{enumitem}
\setlist[enumerate,1]{label=\arabic*.}
\setlist[enumerate,2]{label=(\alph*)}
\setlist[enumerate,3]{label=(\roman*)}


\DeclareMathOperator{\perm}{P}
\DeclareMathOperator{\comb}{C}
\newcommand{\e}{\mathrm{e}}

% To have all the question numbers be odd, we use an enumerate
% environment to number the questions and, for every question after the
% first, we begin a new question with
% \stepcounter{enumi}
% \item


%--------------------------------------------------------------------
\begin{document}

\begin{center}
  \bfseries
  Math 225 Assignment 3 Odd\
  Homework number 3, due Sep ?, 2018\\
  Odd numbered questions
\end{center}

\bigskip

\begin{enumerate}
\item Build a generating function for ${a_r}$, the number of distributions of r identical
objects into:
\begin{enumerate}
 \item Five different boxes with at most five objects in each box.
\begin{displaymath}
g(x) =(1+x+x^2+x^3+x^4+x^5)^5
\end{displaymath}

\bigskip

  \item 3. Build a generating function for ${a_r}$, the number of integer solutions to the equations:
\begin{enumerate}
\item ${e_1 + e_2 + e_3 + e_4 + e_5 = r, 0 ≤ e_i ≤ 4}$
\item ${e_1 + e_2 + e_3  = r, 0 ≤ e_i< 5}$
\item ${e_1 + e_2 + e_3 +e_4 = r, 2 ≤ ei ≤ 6}$ e1 even, e2 odd
\item ${e_1 + e_2 + e_3 +e_4 = r, 0 ≤ ei}$
\item ${e_1 + e_2 + e_3 +e_4 = r, 0 < e_i}$ e2, e4 odd, e4 ≤ 3

\begin{displaymath}
g(x) =(x^3+x^4+x^5+x^6)^4
\end{displaymath}

\item Seven different boxes with at least one object in each box.
\begin{displaymath}
g(x) =(x+x^2+x^3+x^4+x^5+x^6+...+x^n)^7
\end{displaymath}

\item Three different boxes with at most five objects in the first box  
\begin{displaymath}
g(x) =(1+x+x^2+x^3+x^4+x^5)\cdot(1+x+x^2+x^3+....+x^n)^2
\end{displaymath}

  \end{enumerate}
  


  \begin{enumerate}
  \item It depends on what you mean by \emph{buried}.
  \item Perhaps, because $\comb(6,2) = \binom{6}{2}$.
  \end{enumerate}
  Then again, $\perm(6,2) = \frac{6!}{4!}$.  That's worth repeating
  more prominently:
  \begin{displaymath}
    \perm(6,2) = \frac{6!}{4!}
  \end{displaymath}

\stepcounter{enumi}
\item What was the color of George Washington's white horse?

  It was a very pale blue, easily mistaken for white.  Some may doubt
  this, but it's important to remember that
  \begin{itemize}
  \item a pale blue viewed in the reddish light of evening can appear
    white,
  \item the artist had run out of blue paint, and thought he could get
    by with an approximation, and
  \item blue is a nice color.
  \end{itemize}

\stepcounter{enumi}
\item
  \begin{enumerate}
  \item Define the universe.
  \item Give three examples.
  \end{enumerate}

\bigskip

  \begin{enumerate}
  \item   The world is all that is the case.  Thus, the world will not
    only fit into the case, it \emph{is} the case.
  \item
    \begin{enumerate}
    \item It's actually a rather nice case.
    \item We shall give $\comb(3,1) = \binom{3}{1} = \perm(3,1)$
      examples, but we shall not do that here.
    \end{enumerate}
  \end{enumerate}

\stepcounter{enumi}
\item What is $\e^{x}?$

\bigskip

We have
\begin{displaymath}
  \e^{x} = 1 + x + \frac{x^{2}}{2!} + \frac{x^{3}}{3!} + \cdots
\end{displaymath}
which came out large since it was typeset as a display.

\end{enumerate}

\end{document}

%%% Local Variables:
%%% mode: latex
%%% TeX-master: t
%%% End: