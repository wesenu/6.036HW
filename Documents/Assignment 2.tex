\documentclass[12pt]{amsart}
\usepackage{amssymb}

\usepackage{enumitem}
\setlist[enumerate,1]{label=\arabic*.}
\setlist[enumerate,2]{label=(\alph*)}
\setlist[enumerate,3]{label=(\roman*)}


\DeclareMathOperator{\perm}{P}
\DeclareMathOperator{\comb}{C}
\newcommand{\e}{\mathrm{e}}




%--------------------------------------------------------------------
\begin{document}

\begin{center}
  \bfseries
  Math 225 Assignment 2 Odd\\
  Homework number 1, due Sep 21, 2018
\end{center}

\bigskip

1.There are three women and five men who will split up into two four-person
teams. How many ways are there to do this so that there is at least one woman
on each team?
\bigskip

Step 1: Calculate the total number of ways through which to arrange 8 people on 2 teams. There are C(8,4) ways to choose a subset of 4 out of 8 people, but since who is on the other 4-person team is automatically determined by who is on this team, and since these two teams are not distinct, we divide C(8,4) by 2
\bigskip

Step 2: Calculate the combinations where no woman is on one team. When no woman is on one team, the other team would have all three woman plus one of the five man, and therefore there are 5 ways to make a subset where one team has all three women.
\bigskip
\begin{displaymath}
\comb(8,4)\div2- 5 = \binom{8}{4}\div2-5
\end{displaymath}
\bigskip

3.How many ways are there to distribute 5 identical apples and 6 identical pears
to 3 distinct people such that each person has at least one pear?
\bigskip

Step 1: Give a pear to each of the 3 people, and there are 3 pears and 5 apples left. 
\smallskip

Step 2: Distribute the pears, and according to the model of identical object distribution:
\begin{displaymath}
\comb(r+n-1,r)= \comb(3+3-1,3) = \comb(5,3)
\end{displaymath}
Step 3: Distribute the pears, and according to the model of identical object distribution:
\begin{displaymath}
\comb(r+n-1,r)= \comb(5+3-1,5) = \comb(7,5)
\end{displaymath}
Step 4: Using the multiplication rule, the total number of ways is:
\begin{displaymath}
\comb(5,3)\cdot \comb(7,5)
\end{displaymath}

\bigskip
5.How many ways are there to distribute 16 different toys among four children?
\smallskip

(a) Without restrictions?
\smallskip

Using the model for distribution of distinct objects, we have 4 possible owners for each toys, thus
\begin{displaymath}
n^r= 4^{16}
\end{displaymath}
(b) If two children get 6 toys and two children get 2 toys?
Since two children get 6 toys and two children get 2 toys, we can use the model of distinct object distribution, thus, we have 
\begin{displaymath}
\perm(r;r_1,r_2,...,r_n) = \perm (16; 6,6,2,2) = \frac{16!}{6!6!2!2!}
\end{displaymath}

(c) With each child getting 4 toys?
\bigskip

Similarly, we can use the model of distinct object distribution, 
\begin{displaymath}
\perm(r;r_1,r_2,...,r_n) = \perm (16; 4,4,4,4) = \frac{16!}{(4!)^4}
\end{displaymath}
\bigskip

7. How many integer solutions are there to x1 + x2 + x3 + x4 + x5 = 28 with:
\smallskip 

(a) xi greater than or equal to 0
\smallskip

We distribute 28 identical ones among the 5 x, and thus, we can use the model of distinct object distribution,
\begin{displaymath}
\comb(r+n-1,r)= \comb(28+5-1,28) = \comb(32,28) 
\end{displaymath}
\smallskip

(b) xi greater than 0
\smallskip

since all x should be greater than 0, and all x are integers, we assign 1 to each of them, so that x could meet the requirement, than, we use the model of distinct object distribution, 
\smallskip

Step 1: Assign 1 to each of the x, so we have (28-5) left in total.
\smallskip

Step 2:
\begin{displaymath}
\comb(r+n-1,r)= \comb((28-5)+5-1,23) = \comb(27,23) 
\end{displaymath}
\bigskip

9. How many arrangements of the letters in INSTRUCTOR have all of the following
properties simultaneously:
(a) The vowels appearing in alphabetical order.
(b) At least 2 consonants between each vowel, and
(c) Begin or end with the 2 T’s (the T’s are consecutive).
\smallskip

Step 1: Let's see what we have in INSTRUCTOR: (10 char in total)
\smallskip

one I, one O, one U, one N, one S, two Ts, one R, one C,  one R
\smallskip

Step 2: We first satisfy constraint (c) by grouping the two Ts together. We initially have ten characters in the word, and after grouping the two Ts, we have nine objects left. There are 2 ways to place the Ts, either in the beginning or the end. 
\smallskip

Step 3: We can ignore the Ts for this part. Out of the eight chars, three of them are vowels. Since there are two consonants between each vowel, then there are 4 ways to choose the position for the vowels, one where the first vowel in the sequence is placed in the first position with two consonants in between each vowels;  the second where the first vowel is placed in the second position with two consonants in between each vowels, the third where the first vowel in the sequence is placed in the first position with three consonants following it, then a vowel, then finally the 
\smallskip

Step 4: Since the vowels have to be in alphabetical order, there is only 1 way to put the vowels in position. 
\smallskip

Step 5: We arrange the five remaining consonants in the remaining positions. 
\begin{displaymath}
5!
\end{displaymath}
\smallskip

Step 6: We use the multiplication principle.
\begin{displaymath}
5! \cdot 2\cdot 4 ={5!}\cdot 8
\end{displaymath}
\smallskip


\end{document}