\documentclass[12pt]{amsart}
\usepackage{amssymb}

\usepackage{enumitem}
\setlist[enumerate,1]{label=\arabic*.}
\setlist[enumerate,2]{label=(\alph*)}
\setlist[enumerate,3]{label=(\roman*)}


\DeclareMathOperator{\perm}{P}
\DeclareMathOperator{\comb}{C}
\newcommand{\e}{\mathrm{e}}

% To have all the question numbers be even, we use an enumerate
% environment to number the questions and we begin a new question with
% \stepcounter{enumi}
% \item


%--------------------------------------------------------------------
\begin{document}

\begin{center}
  \bfseries
  Ima Student\\
  Homework number 1, due July 4, 1776\\
  Even numbered questions
\end{center}

\bigskip

\begin{enumerate}
\stepcounter{enumi}
\item Who's buried in Grant's tomb?  In particular,
  \begin{enumerate}
  \item Is \emph{anyone} buried in Grant's tomb?
  \item Does the tomb really belong to Grant?
  \end{enumerate}

  Here's some text that's part of the answer. Here's some text that's
  part of the answer. Here's some text that's part of the
  answer. Here's some text that's part of the answer. Here's some text
  that's part of the answer. Here's some text that's part of the
  answer. Here's some text that's part of the answer. Here's some text
  that's part of the answer. Here's some text that's part of the
  answer.
  \begin{enumerate}
  \item It depends on what you mean by \emph{buried}.
  \item Perhaps, because $\comb(6,2) = \binom{6}{2}$.
  \end{enumerate}
  Then again, $\perm(6,2) = \frac{6!}{4!}$.  That's worth repeating
  more prominently:
  \begin{displaymath}
    \perm(6,2) = \frac{6!}{4!}
  \end{displaymath}

\stepcounter{enumi}
\item What was the color of George Washington's white horse?

  It was a very pale blue, easily mistaken for white.  Some may doubt
  this, but it's important to remember that
  \begin{itemize}
  \item a pale blue viewed in the reddish light of evening can appear
    white,
  \item the artist had run out of blue paint, and thought he could get
    by with an approximation, and
  \item blue is a nice color.
  \end{itemize}

\stepcounter{enumi}
\item
  \begin{enumerate}
  \item Define the universe.
  \item Give three examples.
  \end{enumerate}

\bigskip

  \begin{enumerate}
  \item   The world is all that is the case.  Thus, the world will not
    only fit into the case, it \emph{is} the case.
  \item
    \begin{enumerate}
    \item It's actually a rather nice case.
    \item We shall give $\comb(3,1) = \binom{3}{1} = \perm(3,1)$
      examples, but we shall not do that here.
    \end{enumerate}
  \end{enumerate}

\stepcounter{enumi}
\item What is $\e^{x}?$

\bigskip

We have
\begin{displaymath}
  \e^{x} = 1 + x + \frac{x^{2}}{2!} + \frac{x^{3}}{3!} + \cdots
\end{displaymath}
which came out large since it was typeset as a display.

\end{enumerate}

\end{document}

%%% Local Variables:
%%% mode: latex
%%% TeX-master: t
%%% End: